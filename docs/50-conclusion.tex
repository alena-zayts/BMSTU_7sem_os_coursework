\chapter*{Заключение}
\addcontentsline{toc}{chapter}{Заключение}

В ходе выполнения курсовой работы был определен способ перехвата входящих и исходящих пакетов -- путем регистрации функций перехвата с использованием библиотеки Netfilter. В качетсве точек перехвата было решено использовать точку, которую проходят все входящие пакеты (NF\_INET\_PRE\_ROUTING), и точку, которую проходят все исходящие пакеты (NF\_INET\_POST\_ROUTING).

В качестве параметров правил фильтрации пакетов были выбраны протокол передачи и  ip-адреса и порты источника и назначения. 

Были рассмотрены различные виды спуфинг-атак и разработаны меры защиты от них. Для защиты от IP-спуфинга было решено отбрасывать все входящие пакеты с ip-адресом сети защищаемого хоста в качестве ip-адреса источника и все исходящие пакеты с ip-адресом источника, отличным от внутренних адресов сети, независимо от протокола передачи и портов. Для защиты от DNS-спуфинга было решено проверять бит AA поля заголовка DNS-пакета и отбрасывать все ответы от неавторитетных DNS-серверов. 

