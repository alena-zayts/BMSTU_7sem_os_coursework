\section*{ВВЕДЕНИЕ}
\addcontentsline{toc}{section}{ВВЕДЕНИЕ}

Обеспечение сохранности данных в сети является одной из актуальных задач. Для обнаружения и предупреждения сетевых атак используются специальные программные или программно-аппаратные комплексы, ключевым компонентом которых является межсетевой экран.


Межсетевой экран, как правило, представляет собой часть сетевой подсистемы хоста в распределенной системе и реализует функции перехвата, анализа и модификации сетевых пакетов на основе заданной системы правил~\cite{b0}. Разработке межсетевого и будет посвящена данная работа.


%Межсетевые экраны обеспечивают барьер между сетями и предотвращают или блокируют нежелательный или несанкционированный трафик. Единственного определения для межсетевого экрана не существует. В данной работе будем использовать следующее определение межсетевого экрана. Межсетевой экран – система или группа систем, используемая для управления доступом между доверенными и не доверенными сетями на основе предварительно сконфигурированных правил [1]. Межсетевые экраны могут управлять доступом к сети и от нее. Они могут настраиваться для предотвращения получения доступа к внутренним сетям и услугам несанкционированных пользователей. Они могут также конфигурироваться для предотвращения нежелательного доступа к внешним или несанкционированным сетям и услугам внутренних пользователей. МСЭ обеспечивает также выполнение следующих функций:








%Одной из актуальных задач обеспечения безопасности является защита отдельного хоста в распределённой системе  от потенциально вредоносных сообщений. Для этого используются различные программные и аппаратные средства, например, межсетевые экраны.

%Межсетевой экран -- это система или группа систем, используемая для управления доступом между доверенными и недоверенными сетями на основе предварительно сконфигурированных правил \cite{b1}. Разработка межсетевого экрана и будет рассмотрена данная работа.















%Одной из важнейших задач обеспечения безопасности взаимодействия процессов в распределённых системах является задача защиты отдельного хоста от потенциально вредоносных сообщений. 

%Для этого используется специальное программное средство: \textbf{межсетевой экран} (также известный, как сетевой фильтр, брандмауэр, Firewall), разработке которого и посвящена данная работа.


%Одной из актуальных задач является обеспечение сохранности данных в сети. Для того чтобы предотвратить проникновение вредоносного ПО в систему компьютера используются различные программные и аппаратные средства.  Наиболее распространенным средством защиты информации являются сетевые экраны.
%Сетевой экран – система или группа систем, используемая для управления доступом между доверенными и не доверенными сетями на основе предварительно сконфигурированных правил [1]. Сетевые экраны обеспечивают барьер между сетями и предотвращают или блокируют нежелательный или несанкционированный трафик. Для построения сетевого экрана используются определенные методы проверки пакета и различные пакетные фильтры.
