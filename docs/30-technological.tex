\chapter{Технологический раздел}

\section{Выбор средств разработки}

В качестве языка программирования для реализации поставленной задачи был выбран язык Си. Для сборки модуля использовалась утилита make.
%, так как на нем написано ядро OC Linux\cite{ci}
В качестве среды разработки был выбран Qt Creator\cite{qt}, так как он кроссплатформенный, бесплатный и использовался в курсе программирования ранее. 


\section{Инициализация модуля}
В листинге~\ref{lst:init} приведена реализация функции инициализации модуля.

\begin{lstlisting}[caption = {Функция инициализации модуля}, label=lst:init]
int netfilter_init(void)
{
	DMSG("call init");
	
	/* proc-файл для записи правил */
	proc_file = proc_create(PROC_FILE_NAME, S_IRUGO | S_IWUGO, NULL, &proc_fops);
	if (!proc_file) 
	{
		DMSG("call proc_create_data() fail");
		return -ENOMEM;
	}
	DMSG("proc file created");	
	
	/* структура для фильтрации входящих пакетов */
	nfho_in.hook = hook_func_in; // функция, которая будет обрабатывать пакеты
	nfho_in.hooknum = NF_INET_LOCAL_IN; // точка, в которой должна срабатывать функция
	nfho_in.pf = PF_INET; //семейство протоколов
	nfho_in.priority = NF_IP_PRI_FIRST; // приоритет функции (самый высокий)
	
	/* регистрация структуры */
	if (nf_register_net_hook(&init_net, &nfho_in) < 0)
	{
		DMSG("call nf_register_net_hook(NF_INET_LOCAL_IN) fail");
		return -ENOMEM;
	}
	
	/*  структура для фильтрации исходящих пакетов*/
	nfho_out.hook = hook_func_out;
	nfho_out.hooknum = NF_INET_LOCAL_OUT;
	nfho_out.pf = PF_INET;
	nfho_out.priority = NF_IP_PRI_FIRST;
	
	/* регистрация структуры */
	if (nf_register_net_hook(&init_net, &nfho_out) < 0)
	{
		DMSG("call nf_register_net_hook(NF_INET_LOCAL_OUT) fail");
		return -ENOMEM;
	}
	
	DMSG("Firewall module loaded successfully");
	return 0;
}
\end{lstlisting}



\section{Фильтрация сетевых пакетов}

В листинге~\ref{lst:in} приведена реализация функции фильтрации входящих сетевых пакетов.

\begin{lstlisting}[caption = {Функция фильтрации входящих сетевых пакетов}, label=lst:in]
unsigned int hook_func_in(void *info, struct sk_buff *skb, const struct nf_hook_state *state)
{
	int is_unauthorotuve = parse_dns(skb, state->hook);
	if (bu && is_unauthorotuve)
	{
		DMSG("!!! Drop unauthorotive DNS answer");
		return NF_DROP;
	}
	struct iphdr *ip_header;
	sock_buff = skb;
	if(!sock_buff) 
	{ 
		return NF_ACCEPT;
	}
	
	
	ip_header = ip_hdr(sock_buff);
	
	struct firewall_rule cur_rule;
	cur_rule.in = IN;
	snprintf(cur_rule.src_ip, BUFFER_SIZE, "%pI4", &ip_header->saddr);
	snprintf(cur_rule.dest_ip, BUFFER_SIZE, "%pI4", &ip_header->daddr);
	cur_rule.protocol = ip_header->protocol;
	
	if (ip_header->protocol==IPPROTO_UDP)
	{
		struct udphdr *udp_header = (struct udphdr *)skb_transport_header(skb);
		cur_rule.src_port = (unsigned int)ntohs(udp_header->source);
		cur_rule.dest_port = (unsigned int)ntohs(udp_header->dest);
	}
	else if (ip_header->protocol == IPPROTO_TCP)
	{
		struct tcphdr *tcp_header = (struct tcphdr *)skb_transport_header(skb);
		cur_rule.src_port = (unsigned int)ntohs(tcp_header->source);
		cur_rule.dest_port = (unsigned int)ntohs(tcp_header->dest);
	}
	else
	{
		cur_rule.src_port = (unsigned int) 0;
		cur_rule.dest_port = (unsigned int) 0;
	}
	
	if ((bs == 1) && (!strcmp(cur_rule.src_ip, localhost)))
	{
		DMSG("!!!!!! Drop incoming packet as suspicious for spoof, src_ip: %s", cur_rule.src_ip);
		return NF_DROP;
	}
	
	int i;
	for (i = 0; i < in_index; i++)
	{
		if (rules_match(&cur_rule, &(iniplist[i])) == 0)
		{
			DMSG("Drop incoming packet: %s", str_rule(&cur_rule));
			return NF_DROP;
		}
	}
	
	return NF_ACCEPT;
}
\end{lstlisting}




\section{Вывод информации о DNS-пакете}

В листинге~\ref{lst:dns} приведена реализация функции вывода информации о DNS-пакете и проверки того, от какого сервера получен ответ -- авторитетного или нет.

\begin{lstlisting}[caption = {Функция вывода информации о DNS-пакете}, label=lst:dns]
static int show_info_about_dns(struct dnshdr *dns_hdr)
{
	DMSG("DNS %s", dns_hdr->qr == 0 ? "query" : "response"); 
	DMSG("AA bit: %s", dns_hdr->aa == 1 ? "set" : "unset");
	int is_anauthorotive = dns_hdr->aa == 1 ? 0 : 1;
	
	// количество записей в секциях запроса и ответа  
	size_t q_count = ntohs(dns_hdr->qdcount);
	size_t a_count = ntohs(dns_hdr->ancount);
	DMSG("question count = %zd, answer count = %zd", q_count, a_count);
	
	__u8 *section_start = ((__u8 *) dns_hdr) + sizeof(struct dnshdr);
	size_t s_index, s_count = 0;
	
	// секция запросов
	for(s_index = 0; s_index < q_count; s_index++)
	{
		DMSG("question section %zd", s_index + 1);
		s_count = parse_question_section(dns_hdr, section_start);
		section_start += s_count;
	}
	
	//  секция ответов
	for(s_index = 0; s_index < a_count; s_index++)
	{
		DMSG("answer section %zd", s_index + 1);
		s_count = parse_answer_section(dns_hdr, section_start);
		section_start += s_count;
	}
	
	return is_anauthorotive;
}
\end{lstlisting}


